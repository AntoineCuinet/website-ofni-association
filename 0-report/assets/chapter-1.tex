\chapter{Le site web de l'associassion \ofni}
\label{chap:site}

Le but de ce projet est de réaliser le site internet de l'association \ofni.

\section{L'\ofni}
\label{sec:ofni}

Fondée en 1997, l'\ofni\ est l'association des étudiants en informatique de l'\univ\ de \propre{Besançon}.
\bigskip

L'association a pour but de réunir les étudiants des différentes promotions autour d'activités communes (sorties, barbecue, nuit de l'info\ldots) et de permettre une entraide entre promotions.
\bigskip

L'\ofni\ fait également office d’intermédiaire entre les membres de l’association, les chercheurs et les entreprises en conservant un réseau d’anciens membres et des contacts avec nos sponsors.

\section{Présentation du projet}
\label{sec:presentation-projet}

Ce rapport présente le développement que nous avons réalisé dans le câdre du projet tutoré de troisième année de Licence Informatique au sein de l'\univ.
\bigskip

Le sujet de ce projet est de développer un site \web\ pour une association étudiante, en équipe de 2 ou 3 personnes.
Le tout dans l'objectif de mobiliser nos compétences en développement \web\ et d’en acquérir de nouvelles.
Nous avons pour cela été encadrés par M. \nom{Alexandre}{Demougin} et par Mme. \nom{Alicia}{Pierrot}, de la conception à la mise en ligne du site, ainsi que pour la réalisation de ce rapport et de notre soutenance orale.

\section{Objectifs du site}
\label{sec:objectifs-site}

Les consignes sur la réalisation du site étaient de créer une interface claire et intuitive, de développer une solution robuste et fonctionnelle, le tout en mettant l’accent sur l’efficacité et la maintenabilité du site.
\bigskip

Les pistes de fonctionnalités proposées étaient de présenter les diverses activités réalisées par l’association, que le site serve de vitrine pour les projets des étudiants, qu'il permette la réalisation de sondages pour un événement et de pouvoir fournir des informations aux lycéens souhaitant découvrir cette association.

\section{Plan de réalisation du projet}
\label{sec:plan-realisation}

Nous avons eu quelques mois, en parallèle de nos études pour mener à bien ce projet. Celui-ci était divisé en deux grandes phases.
\bigskip

La première, de novembre à décembre 2024, portait sur la phase de conception et de maquettage du site. Celle-ci consistait dans un premier temps à trouver les fonctionnalités du site ainsi que les besoins auxquels il devait répondre. Dans un second temps, cette phase consistait en l'élaboration de la maquette du site, contenant l'entièreté des pages du site ainsi que leurs visuels.
\bigskip

La seconde étape, de janvier à février, était la phase de développement du site. Celle-ci correspond à une période de recherche et de réflexion sur les technologies existantes et le choix des plus adaptées pour la réalisation du projet, ainsi que le développement et le codage du site.
