\chapter{Bilan}

À l’issue de ce projet, nous avons pu livrer une première version du site de l’OFNI, intégrant les principales fonctionnalités définies en amont. 

Ce développement nous a permis de consolider nos compétences en Symfony, en JavaScript et en Sass, tout en découvrant des outils tels que Figma pour le maquettage et Trello pour la gestion de projet.
\bigskip

Si le site répond globalement aux objectifs fixés, certaines fonctionnalités ont dû être adaptées ou reportées en raison de contraintes techniques ou administratives. Malgré cela, nous avons réussi à implémenter une plateforme complète et évolutive, avec une attention particulière portée à la maintenabilité et à l’expérience utilisateur.
\bigskip

Ce projet nous a également permis de mieux appréhender le travail en équipe sur un projet d’envergure, en nous confrontant à des problématiques concrètes de gestion du temps, de répartition des tâches et d’adaptation aux imprévus. 
\bigskip

Nous espérons que ce site constituera une base solide pour les futurs bureaux de l’OFNI et qu’il pourra être enrichi au fil du temps par de nouvelles améliorations.
