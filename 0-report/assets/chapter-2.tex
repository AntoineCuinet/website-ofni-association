\chapter{Le développement du nouveau site}

% TODO: Peut-être revoir où placer la partie répartition du travail, vu qu'on annonce ce qui a été fait par qui, alors que l'on n'a pas encore annoncé ce que l'on voulait faire et ce qui a été implémenté.

% TODO: Peut-être restructurer et mettre la répartition du travail comme une sous-section en début de chaque partie (maquette, développement).

La seconde partie de ce projet a été consacrée à la conception d'une maquette du site, à la recherche et à l’apprentissage des technologies à utiliser, ainsi qu’au développement de la plateforme.

\section{Répartition du travail}

Afin de travailler efficacement en groupe, nous avons réparti la charge de travail de manière équilibrée entre les membres de l'équipe.

Pour la conception de la maquette, nous avons d'abord réfléchi ensemble aux différentes pages qui devaient composer le site, avant que chacun ne prenne en charge certaines d’entre elles. Cette répartition s’est faite naturellement, chacun ayant des préférences différentes.

Lors de la phase de développement, et sur les conseils de nos tuteurs, M. Demougin et Mme Pierrot, nous avons adopté l’outil Trello. Cet outil nous a permis de lister et d’organiser les différentes tâches à accomplir, tout en suivant leur état d'avancement. En collaboration avec nos tuteurs, nous avons ainsi identifié les fonctionnalités à implémenter et avons réparti leur développement au sein du groupe.

Antoine a pris en charge l’implémentation du front-end du site ainsi que le développement du jeu. Il a également rédigé la majorité des contenus textuels du site. Tristan s’est occupé de l’implémentation du système d’ajout et de gestion dynamique des événements et des articles, ainsi que de la création des formulaires d’inscription aux événements. Il a également développé la page boutique. Gaspard s’est quant à lui chargé de la mise en place du système d’authentification et de gestion des comptes, ainsi que de la dynamisation de certaines pages, notamment celles dédiées au jeu et à la présentation de l’association. Il a également développé les formulaires de contact, le système d’envoi automatisé d’e-mails et la page galerie photos.

\section{Le maquettage du nouveau site}

% TODO: Ajouter des images pour illustrer la maquette.

Avant d’entamer le développement, il a été nécessaire de définir précisément les fonctionnalités du site. À travers plusieurs réunions, nous avons réfléchi aux améliorations que nous souhaitions apporter par rapport à l’ancien site de l’OFNI et avons évalué la faisabilité des fonctionnalités envisagées.

Une fois ces choix définis, nous avons entamé la conception de la maquette du site, en deux étapes. Une première version manuscrite nous a permis de structurer le site de manière schématique, puis une version finalisée a été réalisée sur Figma afin d’obtenir une maquette propre et exploitable.

L’objectif principal de cette maquette était de définir les fonctionnalités et leur emplacement, sans se concentrer sur la charte graphique ni sur le design visuel du site.

\subsection{Version manuscrite}

Dans un premier temps, nous avons réalisé une esquisse manuscrite afin d’organiser les différentes pages du site et de positionner les éléments fonctionnels. Cette étape a facilité la réflexion sur la structure générale et l'agencement des fonctionnalités.

% TODO: Ajouter une image d’une page manuscrite de la maquette.

Une fois cette première version terminée, nous avons échangé avec nos tuteurs afin d’ajuster certains éléments. C’est également à ce stade que nous avons commencé à nous interroger sur la conformité de notre site avec le RGPD (Règlement Général sur la Protection des Données).

\subsection{Version finalisée sur Figma}

Après plusieurs ajustements, nous avons élaboré une version numérique définitive sur Figma. Ce logiciel nous a permis de retranscrire nos brouillons manuscrits en une maquette plus aboutie et fidèle à notre vision du projet.

% TODO: Ajouter un aperçu général de Figma et un zoom sur une page spécifique.

\section{Les technologies utilisées}

Une fois la maquette finalisée, nous avons sélectionné les technologies les plus adaptées pour développer un site performant et facilement maintenable par les futurs membres de l’association.

\subsection{Le framework Symfony}

Pour le développement du serveur, nous avons choisi un framework PHP, ayant déjà appris ce langage lors de l’unité d’enseignement \textit{Langages du Web} en deuxième année de licence. Ce choix nous permettait de nous concentrer sur l’architecture du projet sans devoir apprendre un nouveau langage, tout en assurant une continuité pour les futurs membres de l’association qui connaîtront également PHP.

Après avoir comparé Laravel et Symfony, nous avons opté pour ce dernier, son architecture orientée objet étant plus familière et mieux adaptée à notre approche du développement.

\subsection{Le moteur de template Twig}

Pour générer dynamiquement les pages HTML, nous avons utilisé Twig, le moteur de template recommandé par Symfony. Ce choix a facilité la séparation entre la logique métier et l'affichage, améliorant ainsi la lisibilité et la maintenabilité du code.

\subsection{Le langage JavaScript}

Le développement du jeu nécessitait une technologie compatible avec tous les navigateurs modernes. Nous avons donc utilisé JavaScript, qui offre une exécution rapide et une intégration fluide avec le reste du site.

\subsection{Le préprocesseur Sass}

Afin d’améliorer la structuration et la maintenabilité du code CSS, nous avons utilisé le préprocesseur Sass. Cet outil a permis une meilleure organisation des styles et une gestion plus efficace de la mise en page, garantissant une adaptation optimale du site à tous types d’écrans. L’utilisation de Sass facilitera également d’éventuelles évolutions graphiques du site dans le futur.


\section{Les difficultés rencontrées}

