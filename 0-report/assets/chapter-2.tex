\chapter{Le développement du nouveau site}
\label{chap:dev}

% Ne pas lire pour l'instant !!

% TODO: Peut-être revoir où placer la partie répartition du travail, vu qu'on annonce ce qui a été fait par qui, alors que l'on n'a pas encore annoncé ce que l'on voulait faire et ce qui a été implémenté.

% TODO: Peut-être restructurer et mettre la répartition du travail comme une sous-section en début de chaque partie (maquette, développement).

Lors de la première phase du projet, nous avons tout d'abord dû réfléchir aux besoins auxquels le site devait répondre.
À travers plusieurs réunions, nous avons alors réfléchi aux améliorations que nous souhaitions apporter par rapport à l’ancien site de \ofni\ et défini la liste des fonctionnalités que nous souhaitions implémenter.

C'est également à ce moment-là que nous avons dû juger de la faisabilité de certaines fonctionnalités au regard du \logo{RGPD} (Règlement Général sur la Protection des Données).

\subsection{Fonctionnalités du site}
\label{subsec:fonctionnalites}

Ainsi, après ces réunions, voici les fonctionnalités principales que nous avons retenues :

\begin{itemize}
    \item Une page de présentation de l'association et de son histoire
    \item Une page événements et la possibilité de s'inscrire à ces derniers
    \item Une page actualités
    \item Une page boutique permettant d'acheter des produits et d'adhérer à l'association
    \item Une page galerie photos
    \item Un jeu (ajouté au cours de la phase de développement du projet)
    \item Un formulaire de contact pour les entreprises
    \item Une page administrateur pour les futurs bureaux de l'\ofni\ afin de simplifier la transmission de l'association
    \item Un compte utilisateur permettant :
    \begin{itemize}
        \item d'accéder à la galerie photos,
        \item d'accéder au jeu,
        \item de s'inscrire aux événements et de les gérer.
    \end{itemize}
    \item Un compte administrateur permettant :
    \begin{itemize}
        \item de gérer les événements et les actualités,
        \item de gérer les photos,
        \item d'accéder à la page administrateur.
    \end{itemize}
\end{itemize}

\section{Le maquettage du nouveau site}
\label{sec:maquettage}

% TODO: Ajouter des images pour illustrer la maquette.

Une fois ces fonctionnalités définies, nous avons entamé la conception de la maquette du site en deux étapes. Une première version manuscrite nous a permis de structurer le site de manière schématique, puis une version finalisée a été réalisée sur \logo{Figma} afin d’obtenir une maquette propre et exploitable.

L'objectif principal de cette maquette était de confirmer les fonctionnalités désirées et de commencer à visualiser leur implémentation et la forme qu'elles allaient prendre, sans se concentrer sur la charte graphique ni sur le design visuel du site.

\subsection{Version manuscrite}
\label{subsec:maquette-manuscrite}

Dans un premier temps, nous avons réalisé une esquisse manuscrite afin d’organiser les différentes pages du site et de positionner les éléments fonctionnels. Cette étape a été cruciale dans ce projet puisqu'elle nous a facilité la réflexion sur la structure générale du site et l'agencement des fonctionnalités.

C'est également ici que nous avons commencé à imaginer comment rendre notre site le plus maintenable possible sans ajouter de code supplémentaire. Nous avons pensé à créer la possibilité de gérer le contenu du site sans code (comme le ferait un système de gestion de contenu comme \logo{Wordpress} par exemple)
Nous avons également imaginé et schématisé un système de génération de formulaires personnalisés pour les événements.


\begin{figure}[H]
    \centering
    \includegraphics[scale=0.5]{assets/pictures/maquette-inscription.pdf}
    \captionsetup{justification=centering}
    \caption{Page de la maquette manuscrite concernant le système d'inscription à des événements}
    \label{fig:maquette-inscription}
\end{figure}


Une fois cette version manuscrite terminée, nous avons échangé avec nos tuteurs afin d’ajuster certains éléments.
Nous avons notamment été amenés à repenser la façon dont l'utilisateur devait interagir avec le site afin de rendre l'expérience la plus intuitive possible.

\subsection{Version finalisée sur \logo{Figma}}
\label{subsec:maquette-figma}

Après plusieurs ajustements, nous
avons commencé l'élaboration d'une version numérique finalisée sur l'outil \logo{Figma}.
Bien que nous ne connaissions pas le logiciel, nous avons pu rapidement le prendre en main et il nous a ainsi permis de transformer nos brouillons manuscrits en une maquette plus aboutie et fidèle à notre vision du projet.

\begin{figure}[H]
    \centering
    \includegraphics[scale=0.4]{assets/pictures/figma.png}
    \caption{Quelques pages de la maquette numérisée}
    \label{fig:maquette-figma}
\end{figure}

De plus l'apprentissage de \logo{Figma} a également permis à Antoine, qui était chargé du design et de la charte graphique du site, de pouvoir effectuer des maquettes stylisées afin de trouver le visuel qu'il souhaitait appliquer au projet.

\section{Les technologies utilisées}
\label{sec:technologies-utilisees}

Une fois la maquette finalisée, nous avons sélectionné les technologies les plus adaptées pour développer un site performant et facilement maintenable par les futurs membres de l’association.

\subsection{Le \langue{framework} \logo{Symfony}}
\label{subsec:symfony}

Pour le développement du serveur, nous avons choisi un \langue{framework} \logo{PHP}, ayant déjà appris ce langage lors de l’unité d’enseignement \citer{Langages du Web} en deuxième année de licence. Ce choix nous permettait de nous concentrer sur l’architecture du projet sans devoir apprendre un nouveau langage, tout en assurant une continuité pour les futurs membres de l’association qui connaîtront également \logo{PHP}.

Après avoir comparé \logo{Laravel} et \logo{Symfony}, nous avons opté pour ce dernier, son architecture orientée objet étant plus familière et mieux adaptée à notre approche du développement.

\subsection{Le moteur de \langue{template} \logo{Twig}}
\label{subsec:twig}

Pour générer dynamiquement les pages \logo{HTML}, nous avons utilisé \logo{Twig}, le moteur de \langue{template} recommandé par \logo{Symfony}. Ce choix a facilité la séparation entre la logique métier et l'affichage, améliorant ainsi la lisibilité et la maintenabilité du code.

\subsection{Le langage \logo{JavaScript}}
\label{subsec:javascript}

En cours de projet, nous avons été amenés à ajouter une nouvelle fonctionnalité au site : un jeu ludique ayant pour objectif de donner envie aux étudiants de venir sur le site de façon régulière.
L'idée nous a été proposée par un de nos tuteurs, M. \propre{Demougin}, après la conception de la maquette \logo{Figma}.
Cette idée nous a beaucoup plu, et nous avons alors décidé d'implémenter une version revisitée du célèbre jeu \logo{Space Invaders}.

Le langage \logo{JavaScript} a été rapidement adopté pour la réalisation de cette partie du projet, en raison de son intégration facile et de sa compatibilité avec la grande majorité des navigateurs.

\subsection{Le préprocesseur \logo{Sass}}
\label{subsec:sass}

Afin d’améliorer la structuration et la maintenabilité du code \logo{CSS}, nous avons utilisé le préprocesseur \logo{Sass}. Cet outil a permis une meilleure organisation des styles et une gestion plus efficace de la mise en page, garantissant une adaptation optimale du site à tous types d’écrans. L’utilisation de \logo{Sass} facilitera également d’éventuelles évolutions graphiques du site dans le futur.

\section{Répartition du travail}
\label{sec:repartition-travail}

Afin de travailler efficacement en groupe, nous avons réparti la charge de travail de manière équilibrée entre les membres de l'équipe.

Pour la conception de la maquette, nous avons d'abord réfléchi ensemble aux différentes pages qui devaient composer le site, avant que chacun ne prenne en charge certaines d’entre elles. Cette répartition s’est faite naturellement, chacun ayant des préférences différentes.

Lors de la phase de développement, et sur les conseils de nos tuteurs, M. \propre{Demougin} et Mme. \propre{Pierrot}, nous avons adopté l’outil \logo{Trello}. Cet outil nous a permis de lister et d’organiser les différentes tâches à accomplir, tout en suivant leur état d'avancement. En collaboration avec nos tuteurs, nous avons ainsi identifié les fonctionnalités à implémenter et avons réparti leur développement au sein du groupe.

Antoine a pris en charge l’implémentation du front-end du site ainsi que le développement du jeu. Il a également rédigé la majorité des contenus textuels du site.
Tristan s’est occupé de l’implémentation du système d’ajout et de gestion dynamique des événements et des articles, ainsi que de la création des formulaires d’inscription aux événements. Il a également développé la page boutique.
Gaspard s’est quant à lui chargé de la mise en place du système d’authentification et de gestion des comptes, ainsi que de la dynamisation de certaines pages, notamment celles dédiées au jeu et à la présentation de l’association. Il a également développé les formulaires de contact, le système d’envoi automatisé d’\langue{e-mails} et la page galerie photos.

\section{Les difficultés rencontrées}
\label{sec:difficultes}

Plus concrètement, l'implémentation du site se voit confrontée à plusieurs difficultés et interrogations, tant techniques qu'éthiques. Il est attendu de la partie CMS du site, une certaine factorisation. Mais le stockage des données personnelles ainsi que la décision d'autoriser ou non une inscription, sont des questions qui sont à prendre en compte du début à la fin du développement.

\subsection{L'autorisation d'inscrition}
\label{subsec:autorisation-inscription}

Qui peut profiter, et donc s'inscrire, aux événements de l'\ofni~? Ce sont les personnes qui possèdent un compte sur le site internet. Mais la même question revient alors, qui peut créer un compte~? Une décision s'impose alors rapidement~: toute personne membre de l'\univ\ peut se créer un compte, ainsi que certaines autres personnes, comme des alumni par exemple, mais uniquement sur autorisation manuelle du bureau de l'association.

Il faut alors trouver un élément concret qui permet de distinguer un membre de l'\univ\ d'une autre personne. Il est alors logique de se tourner vers les adresses e-mails. En effet, les adresses e-mails des membres de l'\univ\ sont de la forme~: \code{prenom.nom@[edu.]univ-fcomte.fr}.

Lors de la réception d'une nouvelle demande d'inscription, l'algorithme doit donc simplement vérifier que l'adresse \langue{e-mail} de la personne soit conforme au spécimen précédent. Si c'est le cas, alors l'inscription est automatiquement validée à compter de l'instant ou la personne aura utilisé le lien de confirmation d'adresse qui lui a alors été envoyé. Sinon, il doit en faire manuellement la demande via le formulaire de contact situé sur la page d'accueil du site. Qu'importe la voie utilisée, une fois le compte créé, l'utilisateur peut alors s'inscrire aux différents événements.

\subsection{La génération de formulaire}
\label{subsec:generation-formulaire}

L'inscription à un événement nécessite, dans la quasi totalité des cas, le remplissage d'un formulaire par l'utilisateur. Ce dernier étant susceptible de changer à chaque événement, il est donc nécessaire de pouvoir générer dynamiquement ces formulaires en fonction des besoins. Pour cela, l'idée est alors de stocker les formulaires dans la base de données au format \logo{JSON}. Ainsi, lors de la création d'un événement, l'organisateur peut définir le formulaire qui doit être utilisé pour l'inscription. Si ce formulaire n'existe pas déjà, il peut alors le créer. Puis, lors de l'inscription, l'affichage du formulaire côté utilisateur est généré automatiquement en fonction des données stockées.

\subsection{La factorisation des événements}
\label{subsec:factorisation-evenements}

Avoir des formulaires réutilisables est un grand atout pour l'aspect \logo{CMS} du site. Mais au delà de l'inscription, la page utilisée pour présenter chaque événements, rique d'être très similaire d'année en année sur deux éditions d'un même événement. Il est alors nécessaire de pouvoir factoriser ces pages pour éviter de devoir tout recréer tous les texts et toutes les illustrations.
\bigskip

Pour pallier à ce souci de factorisation, la structure du site permet alors à l'administrateur de créer d'abord un événement. Ce dernier est intemporel et n'est donc pas lié à une édition spécifique. Ce dernier centralise un nom, une illustration ainsi qu'un texte de présentation puis un autre texte de résumé. Dans un deuxième temps, l'administrateur peut alors créer une édition de cet événement. Cette édition est alors liée à l'événement et contient les informations spécifiques à son déroulement, comme la date, le lieu, le prix, etc. Ceci est, dans la terminologie du site, appelé une \citer{Instance d'événement}, faisant ainsi une petite référence au monde du développement, qui est après tout le point commun de tous les membres de l'\ofni.

Pour ce qui est de la page qui présente chaque édition d'un événement et fournit le lien d'inscription, elle est alors générée automatiquement à partir des données de l'instance d'événement.
