\chapter*{Résumé}
\label{chap:abstract-fr}

Dans le cadre de notre troisième année de Licence Informatique à l’\univ, nous avons mené un projet tutoré visant à refondre le site \web\ de l’association étudiant \logo{OFNI}. L’objectif était de concevoir et développer un site moderne, ergonomique et maintenable, tout en appliquant nos connaissances en développement \web.

Le projet s’est déroulé en plusieurs phases : un maquettage détaillé à partir des besoins du projet ainsi que le choix des technologies (\logo{Symfony}, \logo{Twig}, \logo{JavaScript}, \logo{Sass}). Ensuite vient le développement du site en lui même. Puis pour finir, la rédaction du présent rapport ainsi qu'une soutenance de projet.

Le site en lui même, intègre des fonctionnalités essentielles, telles qu’une gestion d’événements dynamique, un espace membre, une boutique en ligne et un mini-jeu, \game, visant à encourager la visite du site par les étudiants. Au cours du développement, nous avons rencontré des problématiques techniques et administratives, notamment en ce qui concerne l’authentification des utilisateurs et le respect de la vie privée, encadré par le \logo{RGPD}. Nous avons mis en place un système de gestion flexible des inscriptions et des événements, favorisant la réutilisation des données via une approche factorisée de la gestion de ces derniers.

\section*{Mots-clés}

\noindent Développement web --- Automatisation --- Symfony --- JavaScript --- Sass --- RGPD --- Association

\pagebreak

\chapter*{Abstract}
\label{chap:abstract-en}

As part of our third year in the Computer Science degree program at \univ, we carried out a supervised project aimed at redesigning the \web\ platform of the student association \logo{OFNI}. The objective was to design and develop a modern, ergonomic, and maintainable website while applying our knowledge in \web\ development.

The project was structured into several phases: detailed prototyping based on the project requirements, followed by the selection of technologies (\logo{Symfony}, \logo{Twig}, \logo{JavaScript}, \logo{Sass}). Next came the actual development of the website. Finally, the project concluded with the writing of this report and a project defense.

The website itself includes essential features such as dynamic event management, a member space, an online store, and a mini-game, \game, designed to encourage students to visit the platform regularly. During the development, we encountered technical and administrative challenges, particularly regarding user authentication and privacy compliance, as mandated by the \logo{RGPD}. To address these issues, we implemented a flexible registration and event management system, enabling data reuse through a factorized approach.

\section*{Keywords}

\noindent Web development --- Automation --- Symfony --- JavaScript --- Sass --- RGPD --- Association
