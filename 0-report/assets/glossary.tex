\chapter{Glossaire}

\label{chap:glossary}

\begin{description}
    \item[CMS] Un système de gestion de contenu (\langue{Content Management System}) est un logiciel qui permet de créer, modifier et publier du contenu sur un site \web.
    \item[CAS] Le \langue{Central Authentication Service} est un système d'authentification centralisé permettant à un utilisateur de se connecter à plusieurs services avec un seul identifiant.
    \item[RGPD] Le \citer{Règlement Général sur la Protection des Données} est un règlement de l'Union Européenne qui encadre le traitement des données personnelles.
    \item[Regex] Une expression régulière (\langue{Regular Expression}) est une chaîne de caractères qui décrit un ensemble de chaînes de caractères possibles selon une syntaxe précise.
\end{description}

\noindent\HRule
\vspace{1cm}

\Large\noindent \strong{Terminologie spécifique au site de l'\ofni}\normalsize

\begin{description}
    \item[Instance d'événement] Une occurrence d'un événement, c'est-à-dire une édition précise de ce dernier.
    \item[Form-Widget] Un ensemble de champs de formulaire qui peuvent être réutilisés pour créer des formulaires plus complexes. Il en existe trois types~: \code{natif}, \code{composite} et \code{liste}.
\end{description}
