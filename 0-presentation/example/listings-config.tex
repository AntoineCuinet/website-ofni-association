% définition des couleurs pour la syntaxe
\definecolor{backcolor}{rgb}{0.85, 0.85, 0.85}
\definecolor{codebasic}{rgb}{0.3, 0.3, 0.3}
\definecolor{codeblue}{rgb}{0.2, 0.2, 0.9}
\definecolor{codecyan}{rgb}{0, 0.5, 0.8}
\definecolor{codeorange}{rgb}{1, 0.5, 0}
\definecolor{codegray}{rgb}{0.5, 0.5, 0.5}
\definecolor{codegreen}{rgb}{0, 0.6, 0}

% ajout du PDL++ dans les langages
\lstdefinelanguage{wlang}{
    morekeywords={SI, ALORS, SINON, FIN, POUR, À, TANTQUE, SELON, est, un, une, sont, des, xlsOuvre, xlsFeuilleCourante, xlsDonnée, hExécuteRequête, trierTableau},
    morendkeywords={entier, monétaire, Sources, de, données, chaîne, xlsDocument, xlsLecture, MoiMême, Null},
    sensitive=true,
    morestring=[d]{"},
    morecomment=[l]{//},
    morecomment=[s]{/*}{*/}
}

% définition du style pour les codes sources
\lstdefinestyle{sourcecode}{
    backgroundcolor=\color{backcolor},
    basicstyle=\color{codebasic}\footnotesize\ttfamily,
    identifierstyle=\color{codecyan},
    keywordstyle=\color{codeblue},
    ndkeywordstyle=\color{codeorange},
    stringstyle=\color{codegreen},
    commentstyle=\color{codegray},
    numberstyle=\color{codeblue}\scriptsize\ttfamily,
    captionpos=b,
    numbers=left,
    numbersep=3pt,
    showstringspaces=false,
    breaklines=true,
    breakatwhitespace=true,
    xleftmargin=5pt,
    mathescape=true
}

\lstset{style=sourcecode}

% style pour la frame des codes sources
\newmdenv[linewidth=2, linecolor=codebasic, backgroundcolor=yellow!10, leftmargin=0, rightmargin=0]{sourcecode}